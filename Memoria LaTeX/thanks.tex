\chapter{Agradecimientos}

Mismo ordenador, mismo escritorio y el mismo reto de escribir una página que, para mí, es una de las más <<difíciles>>. Después de alrededor de un año y medio desde que comencé a redactar la del \acs{TFG}, ahora he de hacer lo propio con la de este trabajo.

En primer lugar, agradecer la disposición y dedicación que siempre han mostrado durante todos estos meses ambos directores, tanto Andrés como Luis, puesto que sin su ayuda no podría haber llevado a cabo este \acs{TFM}. Además, quisiera agradecer a todas las personas que componen la \acs{ESI} por el conocimiento que nos han transmitido durante todos estos años, junto con los momentos y amigos que dicha Escuela me ha dado y que me llevo conmigo. También me gustaría agradecer a esos amigos con los que he compartido la experiencia de llegar a hacernos Ingenieros Informáticos, sobre todo a Miguel y José Ángel. Por último, también agradecer a mis otros amigos, los <<de toda la vida>> con los que, aunque no entendían mucho qué era lo que estaba haciendo, he pasado buenos ratos intentando explicarlo.

Finalmente, dejando <<lo mejor>> para el final, agradecer a mi familia el apoyo que he recibido durante todos estos años, especialmente de mis padres. A mi madre, por <<sufrir>> tanto mis buenos como mis malos momentos, ayudarme en lo que podía y por salir adelante en los momentos más difíciles. Y a mi padre, por todo lo que me dio, por mostrarme la puerta que daba a una senda que acabé recorriendo, por enseñarme aptitudes, conocimientos y \textit{hobbies} de los que tomé el relevo y por inculcarme valores que son difíciles de transmitir. Por los momentos que vivimos, que finalmente se han convertido en los recuerdos que más atesoro, como los días de tormenta, aquella <<locura>>, embarrando el coche hasta los amortiguadores o su visión tan clara sobre muchos aspectos acerca del futuro que ahora comienzan a ser una realidad. Una vez, Steve Jobs dijo: <<[...] \textit{aunque algunos los consideren locos, nosotros vemos en ellos a genios. Porque las personas que están lo bastante locas como para creer que pueden cambiar el mundo, son las que lo logran}>>. No sé si llegaría a cambiar algo del mundo, pero de lo que estoy seguro es que cambió mucho del mío.

\quoteauthor{Diego}
