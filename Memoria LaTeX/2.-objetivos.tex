\chapter{Objetivos}
\label{chap:objetivos}

\noindent
\drop{E}{ste} capítulo del trabajo estará dedicado a la exposición tanto del objetivo principal como de los diferentes objetivos parciales o subobjetivos del mismo, así como los posibles problemas o dificultades que, a priori, se pueden vincular con cada uno de ellos.

\section{Objetivo general}
El presente trabajo tiene como objetivo principal la realización de un análisis estratégico enfocado a la definición de un modelo de negocio en el ámbito de la creación de una empresa que ofrezca servicios \acs{DaaS} en un entorno universitario. El proceso a seguir será el de análisis, elección e implantación. Este proceso especifica que, para llevar a cabo el proceso de dirección estratégica, en primer lugar se debe realizar un análisis tanto del entorno general de operación como del específico, elegir una estrategia para desarrollar un modelo de negocio lo más adecuado posible ofreciendo un valor diferencial al cliente y, finalmente, llevar a cabo la implantación final.

El objetivo último de seguir este modelo es el de ofrecer aplicaciones remotas virtualizadas o escritorios virtuales completos a los usuarios teniendo un conocimiento previo del estado del entorno para proporcionar el máximo valor posible al cliente final. De esta manera, se busca facilitar el acceso y utilización de los programas dentro de una organización. Uno de los ejemplos de un servicio de este tipo en un entorno educativo es el de <<Aulas Virtuales>>. Gracias a un servicio de este tipo, los estudiantes que se encuentren en un centro podrían acceder a los recursos requeridos por las diferentes asignaturas sin la necesidad de disponer de ellos en sus ordenadores personales. Por tanto, se facilitaría el seguimiento de las asignaturas y la movilidad, ya que todas las tareas relacionadas, como las de instalación, gestión de licencias o mantenimiento de las aplicaciones, así como las de computación serían llevadas a cabo de manera externa y transparente al usuario. Además, en un intento de llevar a la práctica todo lo anterior, se realizará una prueba de concepto en la que se implemente un prototipo con la finalidad de comprender el funcionamiento y el procedimiento a seguir a la hora de crear y ofrecer un servicio de este tipo.

\clearpage

\section{Objetivos específicos}
A continuación, se expondrán los diferentes objetivos específicos que deberán lograrse para la consecución del objetivo general.

\subsection{Objetivo I: Realizar un análisis del entorno general de operación en pos de identificar factores externos con impacto en la industria objetivo}
Se realizará un análisis del entorno general de operación en el que se desarrollará la actividad de la empresa, puesto que existen determinadas dificultades o fuerzas externas en el entorno empresarial a la hora de desarrollar dicha actividad y operar con normalidad. Ejemplo de esto pueden ser determinados factores derivados de la incertidumbre ante determinados cambios o situaciones, como la situación política o tecnológica.


\subsection{Objetivo II: Realizar un análisis del entorno específico de operación de los servicios \acs{DaaS}}
De igual manera que se debe conocer el entorno general de operación de una empresa, es importante también comprender el entorno específico. Esta tarea se encuentra más enfocada a desarrollar un análisis para asimilar con un mayor grado de profundidad la industria o sector de operación de dicha empresa y los factores que afectan únicamente a todas las empresas localizadas dentro de un mismo sector. Por tanto, el estudio del entorno general y específico será de gran relevancia a la hora de desarrollar y dar forma al modelo de negocio.
%Por otra parte, se realizará un estudio de las alternativas existentes más destacadas en cuanto a términos de \acs{DaaS} se refiere.

\subsection{Objetivo III: Estudio de ventajas competitivas de un servicio \acs{DaaS} para la definición de una estrategia de negocio}
Este estudio de ventajas competitivas se encuentra enfocado a la correcta elección de una configuración apropiada de cara a proporcionar un servicio \acs{DaaS}. De esta manera, se podrá establecer una mejor estrategia de negocio de acuerdo con las ventajas identificadas.

\subsection{Objetivo IV: Definición de un modelo de negocio basado en \acs{DaaS}}
Una vez que se hayan completado los objetivos anteriores y se dispone de una visión global y específica del mercado de operación, junto con las ventajas competitivas, se definirá un modelo de negocio en el que se sustente la operación de la empresa, teniendo en cuenta diferentes elementos como la línea de financiación, gastos o el valor que se ofrecerá finalmente al cliente.

\subsection{Objetivo V: Implementación de un servicio \acs{DaaS} en un entorno \textit{cloud}}
Este último objetivo tendrá como finalidad la implementación a modo de prototipo de un servicio \acs{DaaS} relativo a un caso concreto, donde se recojan los pasos a realizar y se puedan percibir tanto las ventajas como posibles desventajas y dificultades derivadas de su utilización.


% Local Variables:
%  coding: utf-8
%  mode: latex
%  mode: flyspell
%  ispell-local-dictionary: "castellano8"
% End:
