\chapter{Resumen}

<<\textit{Non nova, sed nove}>>, lema de la \acf{URJC}. Del latín, este lema, aplicado al ámbito universitario, significa que no se enseñen cosas nuevas, sino de una manera nueva, diferente. A lo largo de la historia de la humanidad hemos ido obteniendo conocimientos nuevos, ya sea por prueba y error, por inducción o por deducción y las instituciones dedicadas a la enseñanza que se han creado con el tiempo tienen como principal objetivo compartir esos conocimientos, en un intento de que no caigan en el olvido y perduren en el tiempo mejorando, en numerosas ocasiones, la vida de las personas. No obstante, la manera tanto de obtener el conocimiento como de difundirlo no siempre se han realizado de la misma manera, amoldándose a los recursos y las circunstancias en los diferentes momentos históricos.

Este \acf{TFM} se encuentra orientado a explorar una nueva vía de propagación de ese conocimiento, aludiendo a ese <<[..]\textit{sed nova}>> del anterior lema, en un intento de poder llegar a mejorar la calidad de las universidades, constituyendo una nueva forma en la que se apoye la enseñanza, eliminando barreras de entrada y facilitando el seguimiento de los estudios por parte de los alumnos. Por tanto, se estudiará la creación de una nueva empresa que provea de un servicio en la nube de escritorios remotos realizando un análisis estratégico, definiendo un posible modelo de negocio y, finalmente, implementando lo que podría llegar a ser dicho servicio.

\chapter{Abstract}

English version of the previous page.
