\chapter{Resumen}

<<\textit{Non nova, sed nove}>>. Del latín, este lema, aplicado al ámbito universitario, significa que no se enseñen cosas nuevas, sino de una manera nueva, diferente. A lo largo de la historia, el ser humano ha ido obteniendo conocimientos nuevos, ya sea por prueba y error, por inducción o por deducción y las instituciones dedicadas a la enseñanza que se han creado con el tiempo tienen como principal objetivo compartir esos conocimientos, en un intento de que no caigan en el olvido y perduren en el tiempo mejorando, en numerosas ocasiones, la vida de las personas. No obstante, la manera tanto de obtener el conocimiento como de difundirlo no siempre se ha realizado de la misma forma, amoldándose a los recursos y las circunstancias en los diferentes momentos históricos.

Este \acf{TFM} se encuentra orientado a explorar una nueva vía de propagación de ese conocimiento, aludiendo a ese <<[...]\textit{sed nove}>> del anterior lema, con el objetivo de ayudar a mejorar la calidad de la enseñanza en las universidades, constituyendo una nueva forma en la que se apoye dicha enseñanza, eliminando barreras de entrada y facilitando el seguimiento de los estudios por parte de los alumnos, entre otras cosas. Por tanto, se estudiará la creación de una nueva empresa que provea de un servicio en la nube de escritorios remotos realizando un análisis estratégico, definiendo un posible modelo de negocio y, finalmente, implementando lo que podría llegar a ser dicho servicio.

\chapter{Abstract}

"\textit{Non nova, sed nove}". In English, this motto means "\textit{not new things, but in a new way}". Throughout history, human beings have acquired new knowledge, whether by trial and error, by induction or by deduction, and the institutions dedicated to education that have been created over time have the sharing of this knowledge as their main objective, in an attempt to ensure that it does not fall into oblivion and lasts over time, improving people's lives on numerous occasions. However, the way of obtaining and spread knowledge has not always been done in the same way, adapting to resources and circumstances at different historical moments.

This final work aims to explore a new way of spreading this knowledge, alluding to that <<\textit{[...]sed nove}>> of the previous motto, with the aim of contributing to improve the teaching quality in universities, constituting a new form in which teaching can lean on, eliminating entry barriers to access and easing the monitoring of studies, among other things. Taking that into account, the creation of a new company that provides a cloud service of remote desktops performing a strategic analysis, defining a possible business model and lastly implementing how could be a service of this type will be studied.
