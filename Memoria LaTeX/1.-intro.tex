\chapter{Introducción}
\label{chap:introduccion}

\drop{A}{ctualmente}, el mundo es cada vez más digital. Desde el principio de la humanidad, las necesidades de las personas se han ido modificando y amoldando a los recursos disponibles de los que se podía hacer uso. Un ejemplo de esto es el fuego, por medio del cual se comenzó a cocinar y a dar luz y calor al ser humano. Un invento relativamente más cercano podría ser el coche, que sirvió como una herramienta de transporte altamente utilizada hoy en día y que permitía desplazarse sin esfuerzo entre dos puntos o, más recientemente, la creación del ordenador, que permitía realizar cálculos a gran velocidad asistiendo a las personas en trabajos de diferentes áreas. Este último ejemplo es el que más avances ha registrado en un menor tiempo, por lo que se dice que el avance de la informática y la computación ha sido de carácter exponencial. Echando una pequeña vista atrás, el ordenador que llevó al hombre a la Luna hace ahora 50 años disponía de seis veces menos potencia que una calculadora científica de hoy en día y, en consecuencia, mucha menos que un \textit{smartphone} de los que se utiliza a diario por millones de personas \cite{lacaixa2017}. Por ejemplo, hace 50 años la capacidad de cómputo equivalente a la de un simple teléfono móvil podría ocupar una habitación entera mientras que actualmente, es posible llevarla en un bolsillo. Estos avances en la capacidad de cómputo y de las redes que interconectan diferentes ordenadores permiten realizar tareas que antes prácticamente eran impensables, como llevar a cabo predicciones sencillas en poco tiempo, procesar y almacenar ingentes cantidades de datos cada segundo en servidores que se encuentran a cientos de kilómetros de distancia o que otros servidores se encarguen de poner a disposición de los usuarios una gran multitud de servicios de diferentes tipos. Estos servicios se pueden englobar en lo que se denomina como <<computación en la nube>>, que permite prescindir del hecho de disponer de los recursos físicos en un determinado momento, siempre y cuando se disponga de una conexión a Internet.
%La nube ofrece servicios que muchas personas utilizan a diario, como puede ser el correo electrónico, la mensajería instantánea, el almacenamiento de archivos o las copias de seguridad que el \textit{smartphone} realiza cuando se está cargando y en reposo.

\clearpage

No obstante, hasta hace no mucho tiempo, la tendencia era disponer de recursos \textit{on-premise}, es decir, disponer de estos de manera física en las instalaciones, con los gastos adicionales que esta visión conlleva, como el gasto en la inversión inicial y en mantenimiento. Por tanto, se debería realizar un desembolso, muchas veces importante, en la compra de un determinado número de equipos, además de servicios que aseguren la actualización del software asociado y asistencia cuando los usuarios o la empresa lo requiera. Actualmente, lo que se denomina como <<nube>> se ha convertido en uno de los recursos esenciales tanto de las empresas como de los propios usuarios. Según el informe <<\textit{Cloud Computing} en España en 2019>> que ha publicado la consultora Quint Wellington, las inversiones \textit{cloud} del 78\% de las empresas, aproximadamente, aumentarán en los próximos 12 meses \cite{datacentermarket2019}, mientras que para otras como Bosch, la nube es considerada esencial para mejorar sus productos \cite{worldenergytrade2019}. Esto es así puesto que, para las empresas, proporciona valor en tanto que pueden alojar sus archivos, realizar cómputos de manera externalizada o, simplemente, permitir la coordinación de diferentes grupos de trabajo o departamentos internos con el fin de desarrollar su actividad empresarial. Por otra parte, en cuanto a los usuarios, representa un recurso de gran utilidad y valor puesto que ha pasado a ser algo con lo que se trata prácticamente a diario. Ejemplos de algunas funcionalidades son el correo electrónico, el almacenamiento en la nube a través de servicios como Google Drive, Dropbox o iCloud, o soluciones de ofimática \textit{online} como Office 365. Su facilidad de uso, inmediatez y ubicuidad son algunas de las cualidades de estos servicios que hacen de este paradigma algo atractivo de usar.

Además de eso, la nube ofrece multitud de servicios más avanzados de acuerdo a sus tres niveles básicos: \acf{IaaS}, \acf{PaaS} y \acf{SaaS}. El primero se encuentra dedicado a ofrecer servicios de manera externalizada, donde el proveedor se encarga de la gestión física de las máquinas y el cliente debe aportar todo lo relacionado con el software, mientras que el segundo permite el desarrollo de aplicaciones sin requerir la instalación de herramientas adicionales. Por último, el tercer nivel se encuentra enfocado a los usuarios finales, puesto que se ofrece un servicio directamente en la nube, preparado para utilizar. A partir de esta arquitectura clásica se han derivado otras capas o niveles, entre los que se encuentra el llamado \acf{DaaS}, donde se ofrece al usuario un escritorio remoto virtualizado con determinadas aplicaciones preinstaladas listas para ser utilizadas, lo que resulta provechoso en entornos específicos, como el empresarial o el universitario, dando lugar a la aparición de términos como la <<ultramovilidad>>.

\clearpage

Estos numerosos avances están cambiando las necesidades de las que se hablaba anteriormente modificando, a su vez, la manera de trabajar dentro de las empresas, la <<ultramovilidad>> se aplica tanto en estas como en los bancos mediante la virtualización de escritorios \cite{antena3noticias2019}. Este término hace referencia a la capacidad de la que disponen los empleados para trabajar desde cualquier parte con independencia del tipo de dispositivo, ya sea tradicional, como un ordenador, o móvil como una \textit{tablet} o \textit{smartphone}. Por tanto, dentro de la llamada transformación digital de las empresas, este concepto ofrece una serie de ventajas \cite{tecnologiaparatuempresa2019}:

\begin{itemize}
    \item \textbf{Simplicidad}. El empleado solo ha de preocuparse por mantener el dispositivo desde el que se desee conectar.
    \item \textbf{Disponibilidad}. El servicio estará disponible siempre que el empleado lo requiera y siempre que se disponga de conexión a Internet. Por otra parte, todos sus datos y recursos estarán disponibles en la nube en caso de cambio de dispositivo.
    \item \textbf{Agilidad}. Posibilidad de aportar una respuesta rápida a las necesidades del negocio.
    \item \textbf{Seguridad}. Ligado a la disponibilidad, puesto que sus datos se encuentran en la nube, estos permanecerán seguros con independencia del dispositivo.
    \item \textbf{Versatilidad}. Adaptación tanto al entorno fijo como al móvil.
\end{itemize}

De la misma manera que la <<ultramovilidad>> es beneficiosa en el entorno empresarial, en el universitario las ventajas son parecidas puesto que facilita a los interesados el acceso a los materiales y herramientas para el estudio al disponer de todas las aplicaciones en la nube desde cualquier dispositivo, haciendo prescindible la compra de un ordenador capaz de ejecutarlas, además de aumentar la disponibilidad a la hora de realizar cualquier tarea o evitar la compra de licencias del software que se precise. Por otra parte, también aporta una serie de beneficios a la universidad, como evitar disponer de ciertos recursos físicos que un entorno \textit{cloud} podría evitar, así como de su mantenimiento, además de facilitar la entrega de material y servicios docentes tanto a los alumnos como a los profesores a través de un servicio \acs{DaaS}.

Por tanto, debido a que se está comenzando a implantar la virtualización del escritorio en diferentes sectores, el presente \acf{TFM} se encuentra centrado en maximizar el uso de los recursos \textit{cloud}. Para ello, se realizará un estudio estratégico actual en cuanto a servicios \acs{DaaS} concierne en el ámbito de la creación de un modelo de negocio para una nueva empresa que provea de estos a las entidades que lo requieran.

\clearpage

\section{Competencias adquiridas}
La competencias específicas que se han trabajado a lo largo del desarrollo del proyecto son las siguientes:

\begin{itemize}
    \item \textbf{[CE1] Capacidad para la integración de tecnologías, aplicaciones, servicios y sistemas propios de la Ingeniería Informática, con carácter generalista, y en contextos más amplios y multidisciplinares}. Esta primera competencia se ha adquirido gracias a la labor conjunta de investigación y creación de un servicio que provea aplicaciones y escritorios virtuales poniendo el foco en las universidades, que poseen unas necesidades específicas para cada uno de los propósitos y servicios que estas ofrecen a los estudiantes.
    
    \item \textbf{[CE2] Capacidad para la planificación estratégica, elaboración, dirección, coordinación, y gestión técnica y económica en los ámbitos de la ingeniería informática relacionados, entre otros, con: sistemas, aplicaciones, servicios, redes, infraestructuras o instalaciones informáticas y centros o factorías de desarrollo de software, respetando el adecuado cumplimiento de los criterios de calidad y medioambientales y en entornos de trabajo multidisciplinares}. La segunda competencia específica se ha trabajado gracias al análisis estratégico realizado en este \acs{TFM}, la identificación de ventajas competitivas y el desarrollo de un modelo de negocio en el ámbito de la creación de una nueva empresa que provea a las universidades del mencionado servicio de virtualización de aplicaciones y escritorios.
    
    \item \textbf{[CE4] Capacidad para modelar, diseñar, definir la arquitectura, implantar, gestionar, operar, administrar y mantener aplicaciones, redes, sistemas, servicios y contenidos informáticos}. Esta última competencia se ha asimilado durante el trabajo realizado en relación al propio servicio de virtualización, donde se ha interiorizado su proceso de creación, definición de máquinas y su forma de operación, además del desarrollo de un \textit{script} con interfaz gráfica con la intención de facilitar la tarea de gestión de usuarios de dicho servicio.
\end{itemize}

\clearpage

\section{Estructura del documento}

A continuación, se detalla la estructura que posee el presente documento.

\begin{definitionlist}
    \item[Capítulo \ref{chap:introduccion}: \nameref{chap:introduccion}] En este primer capítulo se realiza una breve introducción al problema y se pone en contexto al lector.
    \item[Capítulo \ref{chap:objetivos}: \nameref{chap:objetivos}] Este capítulo se encuentra dedicado a la exposición de los diferentes objetivos de los que se compone el trabajo.
    \item[Capítulo \ref{chap:antecedentes}: \nameref{chap:antecedentes}] El tercer capítulo se encontrará destinado a poner en contexto al lector acerca de lo que representará el núcleo del presente trabajo.
    \item[Capítulo \ref{chap:metodo}: \nameref{chap:metodo}] Se expone la metodología de trabajo que se ha seguido para desarrollar y llevar a cabo los objetivos definidos, así como los diferentes recursos de los que se va a hacer uso.
    \item[Capítulo \ref{chap:resultados}: \nameref{chap:resultados}] En este capítulo se detallan los resultados que se han obtenido como producto de seguir la metodología escogida.
    \item[Capítulo \ref{chap:conclusiones}: \nameref{chap:conclusiones}] El último capítulo se reserva para las conclusiones obtenidas, así como las posibles tareas que puedan representar trabajos futuros en relación a este proyecto.
\end{definitionlist}


% Local Variables:
%  coding: utf-8
%  mode: latex
%  mode: flyspell
%  ispell-local-dictionary: "castellano8"
% End:
