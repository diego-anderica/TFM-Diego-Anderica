\chapter{Conclusiones y Trabajos Futuros}
\label{chap:conclusiones}

\drop{P}{or} último, este capítulo del proyecto se encontrará dedicado a exponer las conclusiones a modo de resumen del trabajo realizado durante estos meses. Asimismo, se presentarán determinados trabajos futuros que pudieran realizarse de cara a una mejora final en el nivel o calidad del servicio ofrecido.

\section{Conclusiones}
Atendiendo a lo descrito en el Capítulo \ref{chap:objetivos} y, más concretamente, al objetivo principal de este \acs{TFM}, se especificaba como tal la realización de un análisis estratégico para la definición de un modelo de negocio de una nueva empresa que ofreciera servicios \acs{DaaS} dirigidos al entorno universitario. A lo largo de este proyecto se ha trabajado el concepto de análisis estratégico, siguiendo sus resultados hasta la definición de un modelo de negocio centrado en el servicio ofrecido. Por tanto, el objetivo principal se considera satisfecho. En cuanto a los objetivos parciales, se consideran igualmente alcanzados ya que estos, en su mayoría, están estrechamente relacionados con el objetivo principal. Se ha realizado un análisis del entorno general de operación, donde se ha investigado y tratado sobre temas que conciernen de manera genérica a las empresas cuya operación se sitúe en el mismo ámbito que el de la que se pretende desarrollar. De igual manera, se ha focalizado posteriormente en el entorno específico, realizando un estudio cuyo radio de acción afecta a la industria analizada, para lo que se ha hecho uso del modelo de las cinco fuerzas de Porter. Además, se han identificado una serie de ventajas competitivas que pudieran resultar de interés para la empresa, lo que serviría de ayuda de cara a posicionarse estratégicamente para así destacar sobre el resto, impulsando su actividad y rentabilidad. Fruto de toda la tarea analítica anterior, se ha llegado a la definición  de un modelo de negocio coherente con lo mencionado y que podría representar una propuesta válida de cara a desarrollar la actividad de la empresa. Por último, se ha realizado la creación de un servicio \acs{DaaS} en un entorno \textit{cloud} (Microsoft Azure), que podría servir como muestra del servicio que la empresa ofrecería a los potenciales clientes.

\clearpage

Además, durante esta última parte del proyecto, se ha trabajado con el personal del \acf{CTIC} de la \acs{UCLM} con el fin de probar y validar el servicio realizando dos casos de uso diferentes, verificando que podría ser de aplicación en un entorno universitario real y comparando los gastos de adquisición de hardware con los del servicio de \acs{WVD}. Por otra parte, se ha comprobado que el modelo de negocio y servicio planteados son capaces de satisfacer las necesidades que podría tener una universidad de la envergadura de la \acs{UCLM} en cuanto a la llamada <<digitalización>> y migración de recursos \textit{on-premise} a la nube.

\section{Trabajos futuros}
Con el fin de que la actividad de la empresa se desarrolle de la manera más eficiente posible, una vez que se encuentre instaurada y realizando su actividad, el hecho de realizar un análisis \acs{DAFO} podría resultar interesante, así como la repetición del análisis del sector con relativa frecuencia, puesto que es posible que pueda variar con el tiempo y condicionar los movimientos de las empresas que se encuentren en él. Dicho análisis \acs{DAFO} podría ser complementario a todo lo realizado en este \acs{TFM}, teniendo en cuenta tanto factores internos (debilidades y fortalezas) como factores externos (amenazas y oportunidades). Además, sería conveniente realizar un plan de negocio, que representa un paso más en la definición de una empresa de manera más formal a medio plazo, así como el desarrollo de una arquitectura empresarial, de manera que se pudiera llevar a cabo un buen gobierno de \acs{TI}. Por otra parte, atendiendo al script que se ha desarrollado, existe un cierto margen de mejora, puesto que se ha orientado específicamente a la gestión de usuarios. Dicho script podría crecer hasta incluir todas las operaciones que permite realizar el servicio en la actualidad, como la creación de nuevos \textit{hostpools} o la definición de políticas de balanceo a través de PowerShell, obviando así la utilización del portal web de Azure. Por último, se podría implementar una máquina virtual que se encargase de la supervisión de los diferentes \textit{hostpools}, de manera que pudiera arrancarlos o detenerlos automáticamente en función de las necesidades en cada momento. Esto supondría una ayuda en las tareas de gestión de dichas máquinas, así como un ahorro económico al no estar haciendo uso de los recursos de Azure si no es necesario.

\newpage