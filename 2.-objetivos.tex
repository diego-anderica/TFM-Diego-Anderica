\chapter{Objetivos}
\label{chap:objetivos}

\noindent
\drop{E}{ste} capítulo del trabajo estará dedicado a la exposición tanto del objetivo principal como de los diferentes objetivos parciales del mismo, así como los posibles problemas o dificultades que pudiera conllevar cada uno de ellos.

\section{Objetivo general}
El presente trabajo tiene como objetivo principal el estudio y definición del conjunto de servicios y recursos \textit{online} (en lo que comúnmente se denomina \textit{cloud} o <<nube>>) que se precisarían a la hora de ofrecer un servicio de <<Aulas Virtuales>>. De esta manera, los estudiantes que se encuentren en un centro podrían acceder a los recursos requeridos por las diferentes asignaturas en curso para hacer uso de estos sin la necesidad de disponer de ellos en cada uno de sus ordenadores personales. Por tanto, de esta manera, se facilitaría el seguimiento de las asignaturas y la movilidad, ya que todas las tareas de computación serían llevadas a cabo de manera externa y transparente al usuario.

\section{Objetivos específicos}
A continuación, se expondrán los diferentes objetivos específicos que deberán lograrse para la consecución del objetivo general.

\subsection{Objetivo I: Estudio y definición de los diferentes roles del sistema}
Al tratarse de un servicio de aulas virtuales enfocado a la docencia y al ámbito educativo, deberán especificarse diferentes roles, donde cada uno de ellos podrá llevar a cabo una serie de acciones determinadas. Además, algunos de estos roles podrían tener impacto sobre el resto de roles o incluso en el sistema en sí.

\subsection{Objetivo II: Estudio y definición de un sistema de autenticación}
Al tratarse de un sistema al que únicamente podrán acceder unos usuarios determinados, se deberá disponer de un sistema de autenticación. De esta manera, dichos usuarios podrán acceder al sistema mediante unas credenciales, asignando unos recursos determinados a cada uno según sus necesidades predefinidas.

\subsection{Objetivo III: Estudio y definición de los recursos necesarios}
Este sistema de aulas virtuales aunará servicios tanto de \acf{IaaS}, como de \acf{PaaS} y \acf{SaaS} en lo que se denomina <<recursos \textit{cloud}>>, por lo que será necesaria la correcta comunicación e integración entre los mismos con el fin de que se satisfaga el objetivo general, entregando valor a los usuarios finales.


% Local Variables:
%  coding: utf-8
%  mode: latex
%  mode: flyspell
%  ispell-local-dictionary: "castellano8"
% End:
